\documentclass{article}[18pt]
%\ProvidesPackage{format}
%Page setup
\usepackage[utf8]{inputenc}
\usepackage[margin=0.7in]{geometry}
\usepackage{parselines} 
\usepackage[english]{babel}
\usepackage{fancyhdr}
\usepackage{titlesec}
\hyphenpenalty=10000

\pagestyle{fancy}
\fancyhf{}
\rhead{Sam Robbins}
\rfoot{Page \thepage}

%Characters
\usepackage{amsmath}
\usepackage{amssymb}
\usepackage{gensymb}
\newcommand{\R}{\mathbb{R}}

%Diagrams
\usepackage{pgfplots}
\usepackage{graphicx}
\usepackage{tabularx}
\usepackage{relsize}
\pgfplotsset{width=10cm,compat=1.9}
\usepackage{float}

%Length Setting
\titlespacing\section{0pt}{14pt plus 4pt minus 2pt}{0pt plus 2pt minus 2pt}
\newlength\tindent
\setlength{\tindent}{\parindent}
\setlength{\parindent}{0pt}
\renewcommand{\indent}{\hspace*{\tindent}}

%Programming Font
\usepackage{courier}
\usepackage{listings}
\usepackage{pxfonts}

%Lists
\usepackage{enumerate}
\usepackage{enumitem}

% Networks Macro
\usepackage{tikz}


% Commands for files converted using pandoc
\providecommand{\tightlist}{%
	\setlength{\itemsep}{0pt}\setlength{\parskip}{0pt}}
\usepackage{hyperref}

% Get nice commands for floor and ceil
\usepackage{mathtools}
\DeclarePairedDelimiter{\ceil}{\lceil}{\rceil}
\DeclarePairedDelimiter{\floor}{\lfloor}{\rfloor}

% Allow itemize to go up to 20 levels deep (just change the number if you need more you madman)
\usepackage{enumitem}
\setlistdepth{20}
\renewlist{itemize}{itemize}{20}

% initially, use dots for all levels
\setlist[itemize]{label=$\cdot$}

% customize the first 3 levels
\setlist[itemize,1]{label=\textbullet}
\setlist[itemize,2]{label=--}
\setlist[itemize,3]{label=*}

% Definition and Important Stuff
% Important stuff
\usepackage[framemethod=TikZ]{mdframed}

\newcounter{theo}[section]\setcounter{theo}{0}
\renewcommand{\thetheo}{\arabic{section}.\arabic{theo}}
\newenvironment{important}[1][]{%
	\refstepcounter{theo}%
	\ifstrempty{#1}%
	{\mdfsetup{%
			frametitle={%
				\tikz[baseline=(current bounding box.east),outer sep=0pt]
				\node[anchor=east,rectangle,fill=red!50]
				{\strut Important};}}
	}%
	{\mdfsetup{%
			frametitle={%
				\tikz[baseline=(current bounding box.east),outer sep=0pt]
				\node[anchor=east,rectangle,fill=red!50]
				{\strut Important:~#1};}}%
	}%
	\mdfsetup{innertopmargin=10pt,linecolor=red!50,%
		linewidth=2pt,topline=true,%
		frametitleaboveskip=\dimexpr-\ht\strutbox\relax
	}
	\begin{mdframed}[]\relax%
		\centering
		}{\end{mdframed}}



\newcounter{lem}[section]\setcounter{lem}{0}
\renewcommand{\thelem}{\arabic{section}.\arabic{lem}}
\newenvironment{definition}[1][]{%
	\refstepcounter{lem}%
	\ifstrempty{#1}%
	{\mdfsetup{%
			frametitle={%
				\tikz[baseline=(current bounding box.east),outer sep=0pt]
				\node[anchor=east,rectangle,fill=blue!20]
				{\strut Definition};}}
	}%
	{\mdfsetup{%
			frametitle={%
				\tikz[baseline=(current bounding box.east),outer sep=0pt]
				\node[anchor=east,rectangle,fill=blue!20]
				{\strut Definition:~#1};}}%
	}%
	\mdfsetup{innertopmargin=10pt,linecolor=blue!20,%
		linewidth=2pt,topline=true,%
		frametitleaboveskip=\dimexpr-\ht\strutbox\relax
	}
	\begin{mdframed}[]\relax%
		\centering
		}{\end{mdframed}}
	
\newcounter{prob}[section]\setcounter{prob}{0}
\renewcommand{\theprob}{\arabic{section}.\arabic{lem}}
\newenvironment{problem}[1][]{%
	\refstepcounter{prob}%
	\ifstrempty{#1}%
	{\mdfsetup{%
			frametitle={%
				\tikz[baseline=(current bounding box.east),outer sep=0pt]
				\node[anchor=east,rectangle,fill=orange!20]
				{\strut Problem};}}
	}%
	{\mdfsetup{%
			frametitle={%
				\tikz[baseline=(current bounding box.east),outer sep=0pt]
				\node[anchor=east,rectangle,fill=orange!20]
				{\strut Problem:~#1};}}%
	}%
	\mdfsetup{innertopmargin=10pt,linecolor=orange!20,%
		linewidth=2pt,topline=true,%
		frametitleaboveskip=\dimexpr-\ht\strutbox\relax
	}
	\begin{mdframed}[]\relax%
	}{\end{mdframed}}
	
% Styling Pseudocode
\lstset{language=C,
	basicstyle=\ttfamily,
	keywordstyle=\bfseries,
	showstringspaces=false,
	morekeywords={if, else, then, print, end, for, do, while, Let},
	tabsize=4,
	mathescape=true,
	escapechar=£,
	numbers=left,
	stepnumber=1,
	frame=top,
	frame=bottom
}

\usepackage{caption}
\DeclareCaptionFormat{listing}{\rule{\dimexpr\textwidth+17pt\relax}{0.4pt}\par\vskip1pt#1#2#3}
\captionsetup[lstlisting]{format=listing,singlelinecheck=false, margin=0pt, font={sf},labelsep=space,labelfont=bf}


% Mathscr font
\usepackage{mathrsfs}

% Ensures there's a bit of the section before a new page
\preto{\subsection}{\Needspace{5\baselineskip}}
\preto{\section}{\Needspace{5\baselineskip}}
\lhead{AZ900}


\begin{document}
\begin{center}
\underline{\LARGE Apply and monitor infrastructure standards with Azure Policy}
\end{center}
\section{IT Compliance}
\begin{definition}[Azure Policy]
	An Azure service to create, assign and manage policies
\end{definition}
RBAC focuses on user actions at different scopes whereas Azure Policy focuses on resouce properties during deployment. Azure Policy is a default allow and explicit deny system, in contrast with RBAC
\subsection{Creating a policy}
To apply a policy you will:
\begin{itemize}
	\item Create a policy definition
	\item Assign a definition to a scope of resources
	\item View policy evaluation results
\end{itemize}
\subsubsection{Policy definition}
A policy definition expresses what to evaluate and what action to take\\
\\
The policy definition itself is represented as a JSON file
\subsubsection{Applying Azure Policy}
This can be done using the Azure portal or one of the command line tools, it takes the following parameters\\
\\
{\renewcommand{\arraystretch}{2}
\begin{tabularx}{\textwidth}{s b}
	\textbf{Parameter}& \textbf{Description}\\
	\hline
	Name & The actual name of the assignment\\
	\hline
	DisplayName& Display name for the policy assignment\\
	\hline
	Definition & The policy definition, based on which you're using to create the assignment\\
	\hline
	Scope& A scope determines what resources or grouping of resources the policy assignment gets enforced on
\end{tabularx}
}
\subsection{Identifying non-compliant resources}
We can use the applied policy definition to identify resources that aren't compliant with the policy assignment through the Azure portal. Similarly, this can be done through the command line
\subsection{Assigning a definition to a scope of resources}
\begin{definition}[Policy assignment]
	A policy definition that has been assigned to take place within a specific scope
\end{definition}
Policy assignments are inherited by all child resources. This inheritance means that if a policy is applied to a resource group, it is applied to that resource group.
\subsection{Policy effects}
{\renewcommand{\arraystretch}{2}
	\begin{tabularx}{\textwidth}{s b}
		\textbf{Policy effect}& \textbf{What happens}\\
		\hline
		Name & The actual name of the assignment\\
		\hline
		Deny& The resource creation/update fails due to policy\\
		\hline
		Disabled& The policy rule is ignored\\
		\hline
		Append& Adds additional parameters/field to the requested resource during creation or update\\
		\hline
		Audit, AuditifNotExists& Creates a warning event in the activity log, but doesn't stop the request\\
		\hline
		DeployIfNotExists& Executes a template deployment when a specific condition is met
	\end{tabularx}
}
\section{Organise policy with initiatives}
\begin{definition}[Initiative definition]
	A set or group of policy definitions to help track your compliance state for a larger goal
\end{definition}
\begin{definition}[Initiative assignment]
	An initiative definition assigned to a specific scope
\end{definition}
\section{Enterprise governance management}
\begin{definition}[Azure Management Groups]
	Containers for managing access, policies and compliance across multiple Azure subscriptions
\end{definition}
Management groups allow you to order your Azure resources hierarchically into collections, which provide a further level of classification that is above the level of subscriptions.
\section{Azure Blueprints}
Azure Blueprints is a declarative way to orchestrate the deployment of various resource templates and other artefacts such as:
\begin{itemize}
	\item Role assignments
	\item Policy assignments
	\item Azure resource manager templates
	\item Resource groups
\end{itemize}
The process of implementing Azure Blueprints consists of the following high-level steps:
\begin{enumerate}
	\item Create an Azure Blueprint
	\item Assign the blueprint
	\item Track the blueprint assignments
\end{enumerate}
The Azure Blueprints service is backed by the Azure Cosmos database to provide low latency and high availability.
\subsection{Comparison with Resource Manager Templates}
Blueprints sit above Resource Manager Templates, also including resource groups, policies and role assignments.\\
\\
Blueprints can be managed directly in Azure, whereas Resource Manager Templates have to be managed separately.
\subsection{Comparison with Azure Policy}
Including a policy in a blueprint enables the creation of the right pattern or design during assignment of the blueprint. 
\section{Compliance Manager}
Microsoft provides resource transparency using the following tools
\subsection{Microsoft Privacy Statement}
This explains what personal data Microsoft processes, how it is processed and for what purpose
\subsection{Microsoft trust center}
This contains all the details about how Microsoft implements and supports security, privacy, compliance and transparency in all Microsoft cloud products.
\subsection{Service Trust Portal}
This hosts the compliance manager service and is the Microsoft public site for publishing audit reports.
\subsection{Compliance Manager}
This provides the following features:
\begin{itemize}
	\item Combines
	\begin{enumerate}
		\item Audit results
		\item Information Microsoft compiles internally for its compliance with regulations
		\item An organisation's self assessment of their own compliance with these standards and regulations
	\end{enumerate}
	\item Enables you to assign, track, and record compliance and assessment-related activities
	\item Provides a compliance score to track progress
	\item Provides a repository for managing evidence relating to compliance
	\item Produces reports
\end{itemize}
\newpage
\section{Monitoring service health}
\subsection{Azure Monitor}
This maximizes the availability and performance of applications using telemetry.
\subsubsection{Data sources}
{\renewcommand{\arraystretch}{2}
	\begin{tabularx}{\textwidth}{L L}
		\textbf{Data Tier}& \textbf{Description}\\
		\hline
		Application monitoring data& Data about the performance and functionality of written code\\
		\hline
		Guest OS monitoring data& Data about the OS your application is running\\
		\hline
		Azure resource monitoring data& Data about the operation of an Azure resource\\
		\hline
		Azure subscription monitoring data& Data about the operation and management of an azure subscription and azure itself\\
		\hline
		Azure tenant monitoring data& Data about the operation of tenant-level Azure services
	\end{tabularx}
}
\subsubsection{Diagnostic Settings}
\textbf{Activity logs} record when resources are created or modified\\
\textbf{Metrics} tell you how the resource is performing and the resources it's consuming\\
\\
You can enable diagnostics
\begin{itemize}
	\item Enable guest-level monitoring
	\item Performance counters - collect performance data
	\item Event logs - Enable various event logs
	\item Crash dumps - enable or disable
	\item Sinks - send your diagnostic data to other services for more analysis
	\item Agent - configure agent settings
\end{itemize}
\subsubsection{Getting more data from your apps}
Application insights - Monitors the availability, performance and usage of your web applications\\
\\
Azure Monitor for containers - Monitors the performance of container workloads\\
\\
Azure Monitor for VMs - Monitors the performance and health of your Windows and Linux VMs
\subsubsection{Responding to alert notifications}
Alerts - Azure notifies you of critical conditions using alerts and can attempt to take corrective actions\\
Autoscale - Ensures you have the right amount of resources running to manage the load on your application effectively
\subsection{Azure Service Health}
This provides guidance when issues with Azure notify you, it is composed of the following views:
\begin{itemize}
	\item Azure Status - Global view of the health state of Azure Services
	\item Service health - Tracks the state of your services in the regions you use them
	\item Resource health - Diagnose and support when an Azure service issue affects your resources
\end{itemize}
\end{document}
